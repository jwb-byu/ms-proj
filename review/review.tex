% setup
\documentclass[11pt]{article}
\usepackage[style=ieee,sorting=none]{biblatex}
\usepackage{etoolbox}
\usepackage{fancyhdr} % For custom headers/footers
\usepackage[a4paper, left=1in, right=1in, top=1in, bottom=1in]{geometry} % left=0.5in, right=0.5in, top=1in, bottom=1in]{geometry}
\usepackage{graphicx}
\usepackage[colorlinks=true, linkcolor=blue, urlcolor=blue, citecolor=blue]{hyperref}
\usepackage{mathptmx} % Use Times New Roman font

% title info
\title{CS 698R Literature Review}
\author{Wesley Borden}
\date{July 8, 2025}

% header/footer
\setlength{\headheight}{14pt}
\pagestyle{fancy}
\fancyhf{} % Clear default header and footer
\fancyhead[L]{CS 698R Literature Review} % Left header
\fancyhead[R]{Wesley Borden} % Right header
\fancyfoot[C]{\thepage} % Center footer with page number
\renewcommand{\headrulewidth}{0pt}

% spacing and indentation
\AtBeginEnvironment{document}{\setlength{\parindent}{0.25in}} % Ensure parindent is set for the document
\newcommand{\sectionwithindent}[1]{
    \section*{#1}
    \hspace{\parindent} % Indent the first paragraph
}
\newcommand{\subsectionwithindent}[1]{
    \subsection*{#1}
    \hspace{\parindent} % Indent the first paragraph
}

% other package setup
\addbibresource{review.bib}

% text
\begin{document}

\sectionwithindent{Data Science is Central to Ongoing Advancements in Neuroscience}
% This intro gives an editorial with some background about why considering data science, etc. together with neuroscience, etc. is relevant and meaningful for both data scientists and neuroscientists.
% Neuroscience needs data science
In 2018, \textit{Nature Methods} published an article stating that ``Neuroscience is experiencing a revolution'' \cite{pandarinath2018autoencoders}. In a field commonly associated with benchtop assays and behavioral outcomes, the paper had little to do with imaging, microscopy, or chemical intervention. Instead, this article introduced a novel computational approach, implementing a neural network-based model to infer functional relationships between active brain cells. As developments in neuroscience continue to unfold, that article is one of many suggesting that computing, automation, statistical analysis, and machine learning will increasingly be at the core of major research achievements and clinical applications in neuroscience.

% Data science has a big opportunity in biology--specifically neuroscience
Since the design of the transformer architecture in 2017 \cite{vaswani2023attentionneed} and commercialization of scaled machine learning in recent years, large language models (LLMs) have rapidly become a ubiquitous technology \cite{wikipedia2025llm}. Though much public attention and industry effort has focused on LLMs and other consumer-facing tools, some of the greatest achievements in machine learning are outside the scope of these applications. For example, machine learning applications have made major contributions to computational biology. In that domain, the AlphaFold algorithms have implemented a model similar to the transformer to increase the number of all known protein structures in the world from $\approx200,000$ to $\approx2,000,000$ \cite{jumper2021alphafold, wikipedia2025alphafold, wikipedia2025pdb}. Just as some of the next decade's greatest achievements in biology and neuroscience will require applications of data science and machine learning, some of the greatest opportunities to innovate with data science and machine learning lie in applications to other sciences, including biology and neuroscience.

% Network science is a specific part of data science that is key for neuroscience and connectomics.
Deep neural network-based models--referred to here as artificial neural networks (ANNs)--are not the only technology showing promise for innovative applications to neuroscience. As neural tissue contains a network of connected, interacting neurons, known as a connectome, network science is similarly applicable. Network science provides a mathematical foundation for modeling connectomes, including excitatory, inhibitory, and modulatory relationships between neurons \cite{emmons2015connectomics, wikipedia2025networkscience, wikipedia2025neuron, wikipedia2025bnn, wikipedia2025connectome}. This foundation includes metrics (e.g., centrality, modularity) and algorithms (e.g., page rank and Louvain) for understanding individual nodes (neurons) and their communities within the network \cite{wikipedia2025networkscience, wikipedia2025pagerank, wikipedia2025louvain}. The foundation of network science is combined with ANNs in graph neural networks (GNNs), including graph autoencoders (GAEs), which involve convolutions on a product of the graph's adjacency matrix, degree matrix, and node feature embeddings to establish structure-aware embeddings of each node \cite{velickovic2018graphattentionnetworks, wikipedia2025gnn}. As modern methodological advances increase capacity for collection of high-resolution connectomes, these network science tools are already beginning to be applied to improve understanding of nervous tissue function (e.g., \cite{srinivasan2025gnnconnectome, neudorf2022gnnconnectome}).

% The plan for this paper is to show synergy between data science and neuroscience subfields
This review considers the intersection of subfields in data science and neuroscience, including connectomics, brain computer interfaces, extracellular electrophysiology, neural signal processing, and spike train analysis. In introducing these subfields, I suggest that they have significant potential for synergy in accelerating understanding of microscale processing and distributed systems in the brain. In addition to providing understanding, I suggest that these same technologies will offer significant new clinical interventions for a wide variety of nervous system diseases and disorders over the next few decades.

\subsectionwithindent{A Connectome Details Neural Connections}
% talk about how this is all based on connectomics. Structure -> Function -> Cognition -> Behavior. This started with Microscopy (Talk about the worm). Discuxx calcium imaging and electrocortiography--recording from pia mater?? .
% Talk about fMRI, dMRI, the HCP, and use of eeg with MRI. That gives rough structural connectivity, but we don't get excit/inhb/modul connections. We just get blood flow. We also don't get stong spatial or temporal resolution.
% A solution to this is in-vivo electrophysiology. It's more invasive than ___, but justified because ___.
% Citations: amunts2013bigbrain, emmons2015connectomics, elam2021hcp, green2015hgp, white1986structure, sejnowski2016nanoconnectomics
Lorem ipsum dolor sit amet consectetur adipiscing elit. Quisque faucibus ex sapien vitae pellentesque sem placerat. In id cursus mi pretium tellus duis convallis. Tempus leo eu aenean sed diam urna tempor. Pulvinar vivamus fringilla lacus nec metus bibendum egestas. Iaculis massa nisl malesuada lacinia integer nunc posuere. Ut hendrerit semper vel class aptent taciti sociosqu. Ad litora torquent per conubia nostra inceptos himenaeos.

Lorem ipsum dolor sit amet consectetur adipiscing elit. Quisque faucibus ex sapien vitae pellentesque sem placerat. In id cursus mi pretium tellus duis convallis. Tempus leo eu aenean sed diam urna tempor. Pulvinar vivamus fringilla lacus nec metus bibendum egestas. Iaculis massa nisl malesuada lacinia integer nunc posuere. Ut hendrerit semper vel class aptent taciti sociosqu. Ad litora torquent per conubia nostra inceptos himenaeos.

\subsectionwithindent{BCI Devices Enable Collection of Partial Functional Micro-Connectomes In-Vivo}
% Talk about BCI's, BrainGate, the Utah Array, Neuralink, Flexible/robotically-inserted fibers, VNS & RNS, DBS & TMS, applications to paralysis, epilepsy, movement disorders, neurodegeneration (ALS), and psychiatric illness (depression-TMS; schizophrenia-DBS)
% citations: card2024neuroprosthesis, geller2018rns, heck2014rns, musk2019integrated, vilela2020bci
% Before we can get into the algorithms, we need to understand the data that we are working with, and how it is collected, as well as potential outputs that are algorithms should have. That starts with an understanding of the hardware that is implanted in the nervous tissue. In the late 2000s the brain gate contortion made significant progress with Braden computer interfaces by implanting devices into the brains of multiple research participants. These individuals were severely paralyzed, and we are implanted with a Utah ray. A Utah ray is in a ray electrodes. The array was implanted like a stamp into their motor cortex, the part of their brain that directly controls muscles like the arms or hands. After being given time to heal, the patients were given control of the computer. Or the device was connected to a computer, such that patient could control the cursor as if they were controlling their hand. Now we want to talk about the algorithms that are between hardware and software. But that is what the output should look like. The brain gate consortium has continued to make progress, for example, recently they were able to do full whole processing, rather than requiring a patient to move a cruiser on the screen the patient could think of a specific words, and the words would appear on the screen or something like that. Simultaneous with the brain gay research has been significant advancement at. The device is implanted with a surgical robot and uses much finer and more flexible, metal strands based on prior imaging of the brain with MRI and CT. The Reva is able to maneuver around blood vessels and other delicate tissue in order to precisely place. The devices electrodes at a higher resolution. This is significantly less invasive for the patient, while maintaining greater penetration or coverage or resolution of the brain tissue. In the early device, about seven patients of received the device to date as well as comprehensive animal studies. Uniquely, the NeuroLink device has built its machine learning models directly based off the voltage data, rather than space Spike sorted data.
% There are other devices that are not typically called brain computer interface, but are doing the same thing. For epilepsy, vagus nerve stimulators are electrical devices that stimulate the vagus nerve, a cranial nerve just outside the brain. It is not a closed loop system, and there is no reading from the brain, but repeated electrical stimulation is given to the third. Nerve. Alter, for treatment, resistant epilepsy, a recurrent nerve stimulator, check the acronym, RNS, reads and writes from a seizure focus, detecting when seizure activity is likely to begin and intervening with inhibitory electrical activity as needed.
% Deep brain stimulation involves open or closed loop stimulation, typically of the basal nuclei. And electrode is inserted surgically, and then controlled from an external device to give repeated simulation or something like that. That and look it up. Site that and look it up. Transcranial magnetic stimulation, involves it device from outside the brain, but uses electromagnetic stimulation to change the brain activity. It has been found effective as a dream at four, severe depression.
% So we have all of these devices and their developing and really in their infancy and say it in doctors are starting to see applications for movement disorders and neural degeneration and psychiatric illness, and the list goes on.
Lorem ipsum dolor sit amet consectetur adipiscing elit. Quisque faucibus ex sapien vitae pellentesque sem placerat. In id cursus mi pretium tellus duis convallis. Tempus leo eu aenean sed diam urna tempor. Pulvinar vivamus fringilla lacus nec metus bibendum egestas. Iaculis massa nisl malesuada lacinia integer nunc posuere. Ut hendrerit semper vel class aptent taciti sociosqu. Ad litora torquent per conubia nostra inceptos himenaeos.

Lorem ipsum dolor sit amet consectetur adipiscing elit. Quisque faucibus ex sapien vitae pellentesque sem placerat. In id cursus mi pretium tellus duis convallis. Tempus leo eu aenean sed diam urna tempor. Pulvinar vivamus fringilla lacus nec metus bibendum egestas. Iaculis massa nisl malesuada lacinia integer nunc posuere. Ut hendrerit semper vel class aptent taciti sociosqu. Ad litora torquent per conubia nostra inceptos himenaeos.

\subsectionwithindent{Extracellular Electrophysiology Optimizes Resolution and Coverage}
% now let's get into how good of a tool electrophysiology is going to be. First: hardware. Before 20__, we had tetrodes and similar rough probes. They could have strong resolution, but poor coverage (NP paper). With the NP probes, now we get strong coverage too. And that's talking about animal research devices. In-clinic, the early 2000 showed the utah array (discussed earlier, cite), but it was hard on the patients because of a rigid array. (electrode arrays). flexible, deep devices, delicately inserted show a lot of potential for maximizing spatial coverage AND spatiotemporal resolution.
% Once the hardware is in: processing on-device vs. offline. Neuralink processes their stuff directly from channels without spike sorting. Makes sense for black-box-style deliverables. Researchers want spike-sorting to increase resolution further. KiloSort leads, but mention competing algorithms. talk about how it's all part of spikeinterface, and talk about the IBL Sorter pipeline.
% citations: ibl2022datarelease, jun2017probes, paulk2022probes
Lorem ipsum dolor sit amet consectetur adipiscing elit. Quisque faucibus ex sapien vitae pellentesque sem placerat. In id cursus mi pretium tellus duis convallis. Tempus leo eu aenean sed diam urna tempor. Pulvinar vivamus fringilla lacus nec metus bibendum egestas. Iaculis massa nisl malesuada lacinia integer nunc posuere. Ut hendrerit semper vel class aptent taciti sociosqu. Ad litora torquent per conubia nostra inceptos himenaeos.

Lorem ipsum dolor sit amet consectetur adipiscing elit. Quisque faucibus ex sapien vitae pellentesque sem placerat. In id cursus mi pretium tellus duis convallis. Tempus leo eu aenean sed diam urna tempor. Pulvinar vivamus fringilla lacus nec metus bibendum egestas. Iaculis massa nisl malesuada lacinia integer nunc posuere. Ut hendrerit semper vel class aptent taciti sociosqu. Ad litora torquent per conubia nostra inceptos himenaeos.

\subsectionwithindent{Scaled Electrophysiology Relies on Strong Neural Signal Processing}
% we need to also have a section about neural signal processing, kilosort, etc.
Lorem ipsum dolor sit amet consectetur adipiscing elit. Quisque faucibus ex sapien vitae pellentesque sem placerat. In id cursus mi pretium tellus duis convallis. Tempus leo eu aenean sed diam urna tempor. Pulvinar vivamus fringilla lacus nec metus bibendum egestas. Iaculis massa nisl malesuada lacinia integer nunc posuere. Ut hendrerit semper vel class aptent taciti sociosqu. Ad litora torquent per conubia nostra inceptos himenaeos.

Lorem ipsum dolor sit amet consectetur adipiscing elit. Quisque faucibus ex sapien vitae pellentesque sem placerat. In id cursus mi pretium tellus duis convallis. Tempus leo eu aenean sed diam urna tempor. Pulvinar vivamus fringilla lacus nec metus bibendum egestas. Iaculis massa nisl malesuada lacinia integer nunc posuere. Ut hendrerit semper vel class aptent taciti sociosqu. Ad litora torquent per conubia nostra inceptos himenaeos.

\subsectionwithindent{Processing Spike Trains Produces a Functional Connectome}
% tons of algorithms exist (cite a bunch. Ranging from wisconsin-madison binning to autoencoders to GLM.). Focus on each of:
% Pairwise Statistics: cross-correlation, transfer entropy, granger causality, GLM
% Also Pairwise, but trained: Autoencoder methods (autoLFADS)
% other new methods I won't replicate: CLDS, NEDS
% citations: pandarinath2018autoencoders, zhang2025neds
Lorem ipsum dolor sit amet consectetur adipiscing elit. Quisque faucibus ex sapien vitae pellentesque sem placerat. In id cursus mi pretium tellus duis convallis. Tempus leo eu aenean sed diam urna tempor. Pulvinar vivamus fringilla lacus nec metus bibendum egestas. Iaculis massa nisl malesuada lacinia integer nunc posuere. Ut hendrerit semper vel class aptent taciti sociosqu. Ad litora torquent per conubia nostra inceptos himenaeos.

Lorem ipsum dolor sit amet consectetur adipiscing elit. Quisque faucibus ex sapien vitae pellentesque sem placerat. In id cursus mi pretium tellus duis convallis. Tempus leo eu aenean sed diam urna tempor. Pulvinar vivamus fringilla lacus nec metus bibendum egestas. Iaculis massa nisl malesuada lacinia integer nunc posuere. Ut hendrerit semper vel class aptent taciti sociosqu. Ad litora torquent per conubia nostra inceptos himenaeos.

\subsectionwithindent{Conclusion}
% breifly mention that we're done talking through details. The big picture is that 20 years from now I anticipate a revolution in medical devices and understanding of neural circutry to be beyond stopping, and similar in scope and size and hype to the modern AI boom. And it will all be powered by data science.
Lorem ipsum dolor sit amet consectetur adipiscing elit. Quisque faucibus ex sapien vitae pellentesque sem placerat. In id cursus mi pretium tellus duis convallis. Tempus leo eu aenean sed diam urna tempor. Pulvinar vivamus fringilla lacus nec metus bibendum egestas. Iaculis massa nisl malesuada lacinia integer nunc posuere. Ut hendrerit semper vel class aptent taciti sociosqu. Ad litora torquent per conubia nostra inceptos himenaeos.

Lorem ipsum dolor sit amet consectetur adipiscing elit. Quisque faucibus ex sapien vitae pellentesque sem placerat. In id cursus mi pretium tellus duis convallis. Tempus leo eu aenean sed diam urna tempor. Pulvinar vivamus fringilla lacus nec metus bibendum egestas. Iaculis massa nisl malesuada lacinia integer nunc posuere. Ut hendrerit semper vel class aptent taciti sociosqu. Ad litora torquent per conubia nostra inceptos himenaeos.

\newpage
\printbibliography

\end{document}
