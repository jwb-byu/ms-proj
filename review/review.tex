% setup
\documentclass[11pt]{article}
\usepackage[style=ieee,sorting=none]{biblatex}
\usepackage{etoolbox}
\usepackage{fancyhdr} % For custom headers/footers
\usepackage[a4paper, left=0.5in, right=0.5in, top=1in, bottom=1in]{geometry}
\usepackage{graphicx}
\usepackage[colorlinks=true, linkcolor=blue, urlcolor=blue, citecolor=blue]{hyperref}
\usepackage{mathptmx} % Use Times New Roman font

% title info
\title{CS 698R Literature Review}
\author{Wesley Borden}
\date{July 8, 2025}

% header/footer
\setlength{\headheight}{14pt}
\pagestyle{fancy}
\fancyhf{} % Clear default header and footer
\fancyhead[L]{CS 698R Literature Review} % Left header
\fancyhead[R]{Wesley Borden} % Right header
\fancyfoot[C]{\thepage} % Center footer with page number
\renewcommand{\headrulewidth}{0pt}

% spacing and indentation
\AtBeginEnvironment{document}{\setlength{\parindent}{0.25in}} % Ensure parindent is set for the document
\newcommand{\sectionwithindent}[1]{
    \section*{#1}
    \hspace{\parindent} % Indent the first paragraph
}
\newcommand{\subsectionwithindent}[1]{
    \subsection*{#1}
    \hspace{\parindent} % Indent the first paragraph
}

% other package setup
\addbibresource{review.bib}

% text
\begin{document}

\sectionwithindent{Modern Data Science is Central to Ongoing Advancements in Neuroscience Research}
% This intro focuses on reviewing achievements in machine learning and data science and considering applications to neuroscience research
In 2018, \textit{Nature Methods} published an article stating that ``Neuroscience is experiencing a revolution\dots'' \cite{Pandarinath2018autoencoders}. In a field commonly associated with benchtop assays and behavioral outcomes, the paper had little to do with imaging, microscopy, or chemical intervention. Instead, this article introduced a novel computational approach, implementing a neural network-based model to infer functional relationships between active brain cells. As developments in the field continue to unfold, that article is one of many that suggest that computing, automation, statistical analysis, and machine learning will increasingly be at the core of major research achievements and clinical applications in neuroscience.

Since the design of the transformer architecture in 2017 \cite{vaswani2023attentionneed} and commercialization of scaled machine learning in recent years, large language models (LLM's) have rapidly become a ubiquitous technology \cite{wikipedia2025llm}. Though much public attention and industry effort has focused on LLM's and other consumer-facing applications, some of the greatest achievements in machine learning are beyond the scope of text generation, image recognition, and media recommendation. For example, in recent years the AlphaFold algorithms have implemented a model similar to the transformer to increase the number of all known protein structures in the world from $\approx200,000$ to $\approx2,000,000$ \cite{Jumper2021,wikipedia2025alphafold,wikipedia2025pdb}. Just as some of the greatest achievements in biology and neuroscience will likely require applications of data and computer science, some of the greatest opportunities to innovate in data and computer science lie in applications to biology and neuroscience.

 Besides advances with large language, models with neural networks, other advances in this area also shall promise. Graph neural networks involve convolutions based on graphs, adjacency, matrix and note in beddings. Adjacent, matrix and note in beddings are input into a matrix multiplication before being passed through the neural network. This gives a new set of embedding. There is an encoding for each note. That is a graph auto and color. With that algorithm, after many rounds of training, the embedding's will converge so that similar nodes, by position and by attribute, will b similar in there in bedding. And bedding clustering can then identify graph partitions or other community structures. Attention networks add to this algorithm by taking into account the relationship between of a nodes neighbors in the convolutions. These graph algorithms, as well as other traditional algorithms and metrics like modularity and partitions and partitioning like in vain. All of those algorithms are useful for working with networks. So if the brain is among the most sophisticated and important networks to exist, how could we apply these algorithms to better understand the brain? First we have to be able to identify the network. Which has not been done in detail for man structures to and to a resolution that is adequate to make definitive conclusions.

In the following sections, we will discuss technologies and tools available to help us in our effort to establish a network relationship based on neural recording data in order to understand and visualize the biological neuronal network.
% TODO: add a bit more about data science, GNNs, GAEs, and GANs. Talk about relevant data science. After that, refine the tone of this all to have a more conservative and cautious tone.
% Here's where I'll put my general comments about AI and ML and Data Science and Graphs Talk about 2017 Transformers, Chatbots, Alphafold, GNN, GAE, and GAN's. The key to this: machine learning has more applications than consumer recommendations and chatbots. Also talk about what I'm building at RHB maybe.

\subsectionwithindent{BCI Devices Interface Directly with Nervous Tissue}
% Talk about BCI's, BrainGate, the Utah Array, Neuralink, Flexible/robotically-inserted fibers, VNS & RNS, DBS & TMS, applications to paralysis, epilepsy, movement disorders, neurodegeneration (ALS), and psychiatric illness (depression-TMS; schizophrenia-DBS)
Before we can get into the algorithms, we need to understand the data that we are working with, and how it is collected, as well as potential outputs that are algorithms should have. That starts with an understanding of the hardware that is implanted in the nervous tissue. In the late 2000s the brain gate contortion made significant progress with Braden computer interfaces by implanting devices into the brains of multiple research participants. These individuals were severely paralyzed, and we are implanted with a Utah ray. A Utah ray is in a ray electrodes. The array was implanted like a stamp into their motor cortex, the part of their brain that directly controls muscles like the arms or hands. After being given time to heal, the patients were given control of the computer. Or the device was connected to a computer, such that patient could control the cursor as if they were controlling their hand. Now we want to talk about the algorithms that are between hardware and software. But that is what the output should look like. The brain gate consortium has continued to make progress, for example, recently they were able to do full whole processing, rather than requiring a patient to move a cruiser on the screen the patient could think of a specific words, and the words would appear on the screen or something like that. Simultaneous with the brain gay research has been significant advancement at. The device is implanted with a surgical robot and uses much finer and more flexible, metal strands based on prior imaging of the brain with MRI and CT. The Reva is able to maneuver around blood vessels and other delicate tissue in order to precisely place. The devices electrodes at a higher resolution. This is significantly less invasive for the patient, while maintaining greater penetration or coverage or resolution of the brain tissue. In the early device, about seven patients of received the device to date as well as comprehensive animal studies. Uniquely, the NeuroLink device has built its machine learning models directly based off the voltage data, rather than space Spike sorted data.

There are other devices that are not typically called brain computer interface, but are doing the same thing. For epilepsy, vagus nerve stimulators are electrical devices that stimulate the vagus nerve, a cranial nerve just outside the brain. It is not a closed loop system, and there is no reading from the brain, but repeated electrical stimulation is given to the third. Nerve. Alter, for treatment, resistant epilepsy, a recurrent nerve stimulator, check the acronym, RNS, reads and writes from a seizure focus, detecting when seizure activity is likely to begin and intervening with inhibitory electrical activity as needed.

Deep brain stimulation involves open or closed loop stimulation, typically of the basal nuclei. And electrode is inserted surgically, and then controlled from an external device to give repeated simulation or something like that. That and look it up. Site that and look it up. Transcranial magnetic stimulation, involves it device from outside the brain, but uses electromagnetic stimulation to change the brain activity. It has been found effective as a dream at four, severe depression.

So we have all of these devices and their developing and really in their infancy and say it in doctors are starting to see applications for movement disorders and neural degeneration and psychiatric illness, and the list goes on.

%%%%%%%%%%%%%%%%%% Pick up here %%%%%%%%%%%%%%%%%% Pick up here %%%%%%%%%%%%%%%%%% Pick up here

Lorem ipsum dolor sit amet consectetur adipiscing elit \cite{white1986structure, emmons2015connectomics}. Quisque faucibus ex sapien vitae pellentesque sem placerat. In id cursus mi pretium tellus duis convallis. Tempus leo eu aenean sed diam urna tempor. Pulvinar vivamus fringilla lacus nec metus bibendum egestas. Iaculis massa nisl malesuada lacinia integer nunc posuere. Ut hendrerit semper vel class aptent taciti sociosqu. Ad litora torquent per conubia nostra inceptos himenaeos.

Lorem ipsum dolor sit amet consectetur adipiscing elit. Quisque faucibus ex sapien vitae pellentesque sem placerat. In id cursus mi pretium tellus duis convallis. Tempus leo eu aenean sed diam urna tempor. Pulvinar vivamus fringilla lacus nec metus bibendum egestas. Iaculis massa nisl malesuada lacinia integer nunc posuere. Ut hendrerit semper vel class aptent taciti sociosqu. Ad litora torquent per conubia nostra inceptos himenaeos.

\subsectionwithindent{A Connectome Elucidates the Relationship between Structure and Function}
% talk about how this is all based on connectomics. Structure -> Function -> Cognition -> Behavior. This started with Microscopy (Talk about the worm). Discuxx calcium imaging and electrocortiography--recording from pia mater?? .
% Talk about fMRI, dMRI, the HCP, and use of eeg with MRI. That gives rough structural connectivity, but we don't get excit/inhb/modul connections. We just get blood flow. We also don't get stong spatial or temporal resolution.
% A solution to this is in-vivo electrophysiology. It's more invasive than ___, but justified because ___.
Lorem ipsum dolor sit amet consectetur adipiscing elit \cite{white1986structure, emmons2015connectomics}. Quisque faucibus ex sapien vitae pellentesque sem placerat. In id cursus mi pretium tellus duis convallis. Tempus leo eu aenean sed diam urna tempor. Pulvinar vivamus fringilla lacus nec metus bibendum egestas. Iaculis massa nisl malesuada lacinia integer nunc posuere. Ut hendrerit semper vel class aptent taciti sociosqu. Ad litora torquent per conubia nostra inceptos himenaeos.

Lorem ipsum dolor sit amet consectetur adipiscing elit. Quisque faucibus ex sapien vitae pellentesque sem placerat. In id cursus mi pretium tellus duis convallis. Tempus leo eu aenean sed diam urna tempor. Pulvinar vivamus fringilla lacus nec metus bibendum egestas. Iaculis massa nisl malesuada lacinia integer nunc posuere. Ut hendrerit semper vel class aptent taciti sociosqu. Ad litora torquent per conubia nostra inceptos himenaeos.

\subsectionwithindent{Electrophysiology-Based Connectomes Have the Greatest Resolution and Coverage}
% now let's get into how good of a tool electrophysiology is going to be. First: hardware. Before 20__, we had tetrodes and similar rough probes. They could have strong resolution, but poor coverage (NP paper). With the NP probes, now we get strong coverage too. And that's talking about animal research devices. In-clinic, the early 2000 showed the utah array (discussed earlier, cite), but it was hard on the patients because of a rigid array. (electrode arrays). flexible, deep devices, delicately inserted show a lot of potential for maximizing spatial coverage AND spatiotemporal resolution.
% Once the hardware is in: processing on-device vs. offline. Neuralink processes their stuff directly from channels without spike sorting. Makes sense for black-box-style deliverables. Researchers want spike-sorting to increase resolution further. KiloSort leads, but mention competing algorithms. talk about how it's all part of spikeinterface, and talk about the IBL Sorter pipeline.
Lorem ipsum dolor sit amet consectetur adipiscing elit \cite{white1986structure, emmons2015connectomics}. Quisque faucibus ex sapien vitae pellentesque sem placerat. In id cursus mi pretium tellus duis convallis. Tempus leo eu aenean sed diam urna tempor. Pulvinar vivamus fringilla lacus nec metus bibendum egestas. Iaculis massa nisl malesuada lacinia integer nunc posuere. Ut hendrerit semper vel class aptent taciti sociosqu. Ad litora torquent per conubia nostra inceptos himenaeos.

Lorem ipsum dolor sit amet consectetur adipiscing elit. Quisque faucibus ex sapien vitae pellentesque sem placerat. In id cursus mi pretium tellus duis convallis. Tempus leo eu aenean sed diam urna tempor. Pulvinar vivamus fringilla lacus nec metus bibendum egestas. Iaculis massa nisl malesuada lacinia integer nunc posuere. Ut hendrerit semper vel class aptent taciti sociosqu. Ad litora torquent per conubia nostra inceptos himenaeos.

\subsectionwithindent{Electrophysiology-Based Connectomes Rely on Spike-Train Analysis}
% tons of algorithms exist (cite a bunch. Ranging from wisconsin-madison binning to autoencoders to GLM.). Focus on each of:
% Pairwise Statistics: cross-correlation, transfer entropy, granger causality, GLM
% Also Pairwise, but trained: Autoencoder methods (autoLFADS)
% other new methods I won't replicate: CLDS, NEDS
Lorem ipsum dolor sit amet consectetur adipiscing elit \cite{white1986structure, emmons2015connectomics}. Quisque faucibus ex sapien vitae pellentesque sem placerat. In id cursus mi pretium tellus duis convallis. Tempus leo eu aenean sed diam urna tempor. Pulvinar vivamus fringilla lacus nec metus bibendum egestas. Iaculis massa nisl malesuada lacinia integer nunc posuere. Ut hendrerit semper vel class aptent taciti sociosqu. Ad litora torquent per conubia nostra inceptos himenaeos.

Lorem ipsum dolor sit amet consectetur adipiscing elit. Quisque faucibus ex sapien vitae pellentesque sem placerat. In id cursus mi pretium tellus duis convallis. Tempus leo eu aenean sed diam urna tempor. Pulvinar vivamus fringilla lacus nec metus bibendum egestas. Iaculis massa nisl malesuada lacinia integer nunc posuere. Ut hendrerit semper vel class aptent taciti sociosqu. Ad litora torquent per conubia nostra inceptos himenaeos.

\subsectionwithindent{Conclusion}
% breifly mention that we're done talking through details. The big picture is that 20 years from now I anticipate a revolution in medical devices and understanding of neural circutry to be beyond stopping, and similar in scope and size and hype to the modern AI boom. And it will all be powered by data science.
Lorem ipsum dolor sit amet consectetur adipiscing elit \cite{white1986structure, emmons2015connectomics}. Quisque faucibus ex sapien vitae pellentesque sem placerat. In id cursus mi pretium tellus duis convallis. Tempus leo eu aenean sed diam urna tempor. Pulvinar vivamus fringilla lacus nec metus bibendum egestas. Iaculis massa nisl malesuada lacinia integer nunc posuere. Ut hendrerit semper vel class aptent taciti sociosqu. Ad litora torquent per conubia nostra inceptos himenaeos.

Lorem ipsum dolor sit amet consectetur adipiscing elit. Quisque faucibus ex sapien vitae pellentesque sem placerat. In id cursus mi pretium tellus duis convallis. Tempus leo eu aenean sed diam urna tempor. Pulvinar vivamus fringilla lacus nec metus bibendum egestas. Iaculis massa nisl malesuada lacinia integer nunc posuere. Ut hendrerit semper vel class aptent taciti sociosqu. Ad litora torquent per conubia nostra inceptos himenaeos.

\newpage
\printbibliography

\end{document}
