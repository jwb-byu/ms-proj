% setup
\documentclass[11pt]{article}
\usepackage{etoolbox} % For patching commands
\usepackage{fancyhdr} % For custom headers/footers
\usepackage[a4paper, left=1in, right=1in, top=1in, bottom=1in]{geometry}
\usepackage{graphicx}
\usepackage[colorlinks=true, linkcolor=blue, urlcolor=blue, citecolor=blue]{hyperref}
\usepackage{mathptmx} % Use Times New Roman font
\usepackage{setspace}
\usepackage{tabularx}
\usepackage{wrapfig}

% title info
\title{CS 698R Project Report}
\author{Wesley Borden}
\date{May 1, 2025}

% header/footer
\setlength{\headheight}{14pt}
\pagestyle{fancy}
\fancyhf{} % Clear default header and footer
\fancyhead[L]{CS 698R Project Report} % Left header
\fancyhead[R]{Wesley Borden} % Right header
\fancyfoot[C]{\thepage} % Center footer with page number
\renewcommand{\headrulewidth}{0pt}

% spacing and indentation
\doublespacing
\AtBeginEnvironment{document}{\setlength{\parindent}{0.25in}} % Ensure parindent is set for the document
\newcommand{\sectionwithindent}[1]{
    \section*{#1}
    \hspace{\parindent} % Indent the first paragraph
}
\newcommand{\subsectionwithindent}[1]{
    \subsection*{#1}
    \hspace{\parindent} % Indent the first paragraph
}

\begin{document}
\subsectionwithindent{Graph Data Science to Improve Electrophysiology Data Analysis for Connectomics}
% Intro
In this project I \href{https://github.com/jwb-byu/ms-proj/blob/main/proposal/drafts/proposal_2025-05-03.pdf}{aimed} to gain an introductory understanding of subfields at the intersection of data science and neuroscience, including neural signal processing, electrophysiology data processing, and general principles in micro-connectomics. I also aimed to understand and replicate existing algorithms to build a cellular-resolution network of a biological neural network. Deliverables are included in the project \href{https://github.com/jwb-byu/ms-proj}{repository}, and here I briefly reflect on highlights and key takeaways from the project.

% Scope and scale of the problem and solutions
As I prepared for this project I anticipated the significance of this interdisciplinary space. However, I don't know that I anticipated how significant recent advances have been. Specifically, I found that multiple advanced methods in statistics and data science have been applied to these problems, including approaches from information theory, systems theory, graph theory, and deep learning. I found that there was much more content in the math applied through these methods than I could learn or replicate in this project, and I saw directions that I could work in to build on this project in the future. At one point in the project Dr. Goodrich and I had a conversation about how the more we learn the more we realize we don't know--so we should seek to become more humble as we grow. I found that as a meaningful soft skill I plan to carry through my career.

% BFS vs. DFS research
In another conversation with Dr. Goodrich, he recommended that I emphasize depth-first over breadth-first research, meaning that as I see sizeable scope for potential research projects, I should focus a niche and lean on others' work. As one who is naturally inclined to understand details, this is an important learning for me. I began to develop discretion for details that were important for me to understand and details where I could rely on others' specialties. This was especially relevant for the more recent signal-processing and spike analysis methods I studies, where I could see that the methods were built on many previously characterized concepts.

% A mathematical pattern to the brain's network
In preparation for this project I worked with Dr. Goodrich in CS 575 to build a relevant foundation of network science. As we did, I began to find that the biological network in neural tissue was organized in a pattern that could be described with data science metrics. The network produced in this project validated that prior learning. Specifically, I \href{https://github.com/jwb-byu/ms-proj/blob/main/replication/replication.ipynb}{found} that biological neuronal networks have strong core-periphery structure, and possibly satisfy the small-world property (more samples are needed to validate), but that within a given anatomical structure there is little organization by centrality or community.

% Conclusion
This project was an excellent opportunity for me to build my background in a topic I am passionate about. I look forward to building on these technical, practical, and soft skill learnings with other opportunities in the future.

\newpage
\subsectionwithindent{Hours}

\begin{tabular}{l r}
    \hline
    \textbf{Category} & \textbf{Time (HH:MM)}\\
    \hline
    Study projects, methods, and code; adapt code for my use & 47:30\\
    Study publications and write literature review & 38:00\\
    Build Demo of IBL Data & 12:45\\
    Study and replicate spike train analysis algorithms & 18:45\\
    Maintain repository and environment & 9:15\\
    Review status periodically & 4:15\\
    Produce report and presentation & 3:30\\
    \hline
    \textbf{Total} & \textbf{134:00}\\
    \hline
\end{tabular}

\end{document}
